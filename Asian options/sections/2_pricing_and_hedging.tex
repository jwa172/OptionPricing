\newcommand{\F}{\mathcal{F}}
\newcommand{\tP}{\widetilde{\mathbb{P}}}
\newcommand{\tW}{\widetilde{W}(t)}
\newcommand{\E}{\mathbb{E}}
\newcommand{\tE}{\widetilde{\E}}

\label{Valuation Problem for Asian Options}

Let \((\Omega, \F, (\F(t))_{t \geq 0}, \tP)\) be the filtered probability space with risk-neutral measure \(\tP\) with the given filtration \((\F(t))_{t \geq 0}\) generated by the Brownian motion \(\tW\).
Defining \(Y(t) = \int_0^t S(t) dt\), the payoff at time \(T\) is given by
\begin{equation}
    V(T) = \left( \frac{1}{T} Y(T) - K\right)^+. 
\end{equation}
The risk-neutral pricing formula in equation \eqref{eq:3} cannot invoke the Markov property with $t$ and $S(t)$.
However, \(S(t)\) is a solution to equation \eqref{eq:1} and $Y(t)$ is a solution to $dY(t) = S(t) dt$ with $Y(0) = 0$, so the two-dimensional process $(S(t), Y(t))$ is a Markov process.
Hence, now the risk-neutral pricing formula in equation \eqref{eq:3} may now invoke the Markov property, where the price is given by a function $v(t, x, y)$ such that
\begin{equation}
    v(t, S(t), Y(t))=\tE\left[e^{-r(T-t)} V(T) \mid \mathcal{F}(t)\right], \quad 0 \leq t \leq T. \label{eq:5}
\end{equation}
Since equation \eqref{eq:5} shows that the discounted option price \(e^{-rt}v(t, S(t), Y(t))\) is a \(\tP\)-martingale, we apply Itò's formula to the discounted option process and see that drift term has to equate to zero. In particular, the price is shown to satisfy the Asian call option PDE \cite{shreve2004stochastic}
\begin{equation}
    r v(t, x, y) = v_t(t, x, y) + r x v_x(t, x, y) + x v_y(t, x, y) + \frac{\sigma^2 x^2}{2} v_{xx}(t, x, y), \label{eq:6}
\end{equation}
where $0<t\leq T, \enspace x \geq 0, \enspace y \in \mathbb{R}$ and boundary conditions
\begin{align*}
    v(t, 0, y) & =e^{-r(T-t)}\left(\frac{y}{T}-K\right)^{+}, 0 \leq t<T, y \in \mathbb{R}, \\
\lim _{y \downarrow-\infty} v(t, x, y) & =0,0 \leq t<T, x \geq 0, \\
v(T, x, y) & =\left(\frac{y}{T}-K\right)^{+}, x \geq 0, y \in \mathbb{R}.
\end{align*}
A short position in the Asian call is hedged by equating \(d(e^{-rt}X(t)) = d(e^{-rt} v(t, S(t), Y(t)))\). This obtains the amount of the underlying asset to hold, $\Delta(t)$, i.e. the delta-hedging formula
\begin{equation*}
    \Delta(t) = v_x(t, S(t), Y(t)).
\end{equation*}
Alternatively, Shreve prices an Asian call that may be averaged a smaller time frame $c \leq T$, with the payoff
\begin{equation*}
    V(T) = \left(\frac{1}{c} \int_{T-c}^{T} S(t) \, dt - K\right)^+,\quad 0 < c \leq T.
\end{equation*}
Firstly, he creates a portfolio process $X(t)$
that holds $\gamma(t)$ shares of the underlying, given by
\begin{equation*}
    \gamma(t) = 
\begin{cases} 
    \frac{1}{rc}(1 - e^{-rc}), & 0 \leq t \leq T - c, \\[8pt]
    \frac{1}{rc} \left(1 - e^{-r(T - t)}\right), & T - c \leq t \leq T,
\end{cases}        
\end{equation*}
and terminal condition $X(T) = \frac{1}{c}\int_{T-c}^T S(t)\ dt $.
Shreve \cite{shreve2004stochastic} derives $X(t)$ in its explicit form to be
\begin{align*}
    X(t) = \frac{1}{rc} \left(1 - e^{-r(T - t)}\right) S(t) 
+ e^{-r(T - t)} \frac{1}{c} \int_{T - c}^{t} S(u) \, du - e^{-r(T - t)} K, \quad T - c \leq t \leq T.
\end{align*}
In particular, \(X(T) = \frac{1}{c} \int_{T-c}^{T} S(t) \, dt - K\).
Secondly, by change of numèraire to $Y(t) = \frac{X(t)}{S(t)}$ and Girsanov's theorem, Shreve creates a new probability measure $\tP^{S}$ \cite{shreve2004stochastic} where
\begin{equation*}
Z(t) = \frac{e^{-rt}S(t)}{S(0)},\quad\forall A \in \F\ \ d\tP^S(A) = \int_A Z(T) d\tP,
\end{equation*}
for which our numèraire $Y(t)$ is a $\tP^S$-martingale. Thus, our risk-neutral pricing formula, as shown by Shreve \cite{shreve2004stochastic} is of the form
\begin{align*}
    V(t) &= e^{rt} \tE \left[ e^{-rT} X^{+}(T) \mid \mathcal{F}(t) \right] \\
&= \frac{S(t)}{e^{-rt} S(t)} \tE \left[  e^{-rT} S(T) \left(\frac{e^{-rT}X(T)}{e^{-rT}S(T)}\right)^{+} \mid \mathcal{F}(t) \right] \\
&= \frac{S(t)}{Z(t)} \tE \left[ Z(T) Y^{+}(T) \mid \mathcal{F}(t) \right]\\
&= S(t) \tE^{S} \left[ Y^{+}(T) \mid \mathcal{F}(t) \right].
\end{align*}
By the Markov property, there exists a function $g(t, y)$ such that \(g(t, Y(t)) = \tE^S \left[ Y^{+}(T) \mid \mathcal{F}(t) \right]\). Consequently, the risk-neutral pricing formula can be expressed as
\begin{equation}
    V(t) = S(t)g(t, Y(t)),
\end{equation}
where $g(t, y)$ is the solution to the PDE
\begin{equation}
    g_t(t, y) + \frac{1}{2} \sigma^2 \left( \gamma(t) - y \right)^2 g_{yy}(t, y) = 0,\quad g(T, y) = y^+,\quad 0 \leq t < T, \, y \in \mathbb{R}, \label{eq:8}
\end{equation}
Both \(v(t, x, y)\) in PDE equation \eqref{eq:6} and \(g(t, y)\) in PDE equation \eqref{eq:8} are theoretical solutions to the arithmetic Asian option pricing problem. However, since the given PDEs cannot be solved analytically, the following sections focus on determining this price through alternative characterizations and numerical methods.